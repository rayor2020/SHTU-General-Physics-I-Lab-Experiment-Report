\documentclass[12pt,a4paper]{article}

% 导入必要的宏包
\usepackage{ctex}           % 支持中文
\usepackage{amsmath}        % 数学公式支持
\usepackage{amsfonts}       % 数学字体
\usepackage{amssymb}        % 数学符号
\usepackage{graphicx}       % 图形支持
\usepackage{float}          % 浮动体控制
\usepackage{booktabs}       % 美观的表格线条
\usepackage{multirow}       % 表格中跨行单元格
\usepackage{geometry}       % 页面设置
\usepackage{fancyhdr}       % 页眉页脚
\usepackage{cite}           % 参考文献引用
\usepackage{pgfplots}

% 设置页面边距
\geometry{left=3.18cm,right=3.18cm,top=2.54cm,bottom=2.54cm}

% 设置标题、作者和日期
\title{弹簧谐振子周期测量实验数据记录与结果}
\author{袁子强\hspace{1cm} 2025533009}
\date{\today}

\begin{document}

% 生成标题
\maketitle

\section{用新型焦利秤测定弹簧劲度系数$K$}

\subsection{原始数据}

\begin{table}[H]
    \center
\begin{tabular}{|c|c|c|c|c|c|}
\hline
砝码质量$m$ ($10^{-3}$kg) & 0 & 1 & 2 & 3 & 4 \\ 
\hline
弹簧长度$y_n$ ($10^{-3}$m) & 327.30 & 332.00 & 338.60 & 343.10 & 349.00 \\
\hline
砝码质量$m$ ($10^{-3}$kg) & 5 & 6 & 7 & 8 & 9 \\
\hline
弹簧长度$y_n$ ($10^{-3}$m) & 354.80 & 360.80 & 366.04 & 371.70 & 377.16 \\
\hline
\end{tabular}
\end{table}

\begin{table}[H]
    \center
\begin{tabular}{|c|c|c|c|c|c|}
\hline
砝码质量$m$ ($10^{-3}$kg) & 9 & 8 & 7 & 6 & 5 \\ 
\hline
弹簧长度$y_n$ ($10^{-3}$m) & 377.16 & 371.80 & 366.40 & 360.60 & 354.88 \\
\hline
砝码质量$m$ ($10^{-3}$kg) & 4 & 3 & 2 & 1 & 0 \\
\hline
弹簧长度$y_n$ ($10^{-3}$m) & 349.40 & 343.50 & 337.50 & 332.18 & 326.98 \\
\hline
\end{tabular}
\end{table}

$F=K\cdot\Delta y$,$\Delta y$即$(y_n-y_0)$,作$F-\Delta y$拟合直线,斜率即为$K$。上海地区$\text{g}=9.794\text{N}/\text{kg}$

\subsection{数据处理}

线性拟合结果:$F = 1788.7\Delta y$,相关系数$r = 0.941$。

劲度系数: $K = 1788.7\pm 290.8\text{N/m}$。

图见下页。

\begin{tikzpicture}
    \begin{axis}[
        width=12cm,
        height=8cm,
        title={弹簧劲度系数测量数据拟合},
        xlabel={伸长量 $\Delta y$ ($10^{-3}$ m)},
        ylabel={拉力 $F$ ($10^{-3}$ N)},
        grid=major,
        legend pos=north west,
        scatter/classes={a={mark=*,red}}
    ]
    
    % 加载过程数据点 (红色)
    \addplot[scatter,only marks,scatter src=explicit symbolic]
    coordinates {
        (0.16, 0)[a]
        (4.86, 9794)[a]
        (11.46, 19588)[a]
        (15.96, 29382)[a]
        (21.86, 39176)[a]
        (27.66, 48970)[a]
        (33.66, 58764)[a]
        (38.90, 68558)[a]
        (44.56, 78352)[a]
        (50.02, 88146)[a]
        (50.02, 88146)[a]
        (44.66, 78352)[a]
        (39.26, 68558)[a]
        (33.46, 58764)[a]
        (27.74, 48970)[a]
        (22.26, 39176)[a]
        (16.36, 29382)[a]
        (10.36, 19588)[a]
        (5.04, 9794)[a]
        (-0.16, 0)[a]
    };
    \addlegendentry{数据}
    
    % 线性拟合直线
    \addplot[domain=-5:55, thick, black, dashed] {1788.7 * x};
    \addlegendentry{拟合直线 $F = 1788.7\Delta y$}
    
    % 在图中添加拟合方程和相关系数
    \node [anchor=south east, draw, fill=white, rounded corners] at (axis description cs:0.97,0.05) {
        \begin{minipage}{3cm}
            \scriptsize
            相关系数: $r = 0.941$
        \end{minipage}
    };
    
    \end{axis}
\end{tikzpicture}

\subsection{误差分析}

加载和卸载曲线不重合,存在明显迟滞,导致数据分散度较大。

系统误差可能包括弹簧质量、空气阻力等因素的影响。

\section{测量弹簧简谐振动周期,计算得出弹簧的劲度系数$K$}

\subsection{原始数据}

\begin{table}[H]
\center
\begin{tabular}{|c|c|c|c|c|c|}
\hline
次数 & 1 & 2 & 3 & 4 & 5 \\ 
\hline
10T & 7.610 & 7.617 & 7.600 & 7.605 & 7.610 \\
\hline
次数 & 6 & 7 & 8 & 9 & 10 \\
\hline
10T & 7.626 & 7.616 & 7.607 & 7.607 & 7.608 \\
\hline
\end{tabular}
\end{table}

其中悬挂砝码质量$M=19$g,弹簧质量$M_0=14.1$g。

\subsection{数据处理}
\begin{itemize}
    \item 求$\overline{T}, \sigma_{\overline{T}}$。

$\overline{T}=0.76106$s,$\sigma_{\overline{T}}=0.000797$s\\
$\therefore T=0.76106\pm 0.00080$s。
    \item 由$T=2\pi\sqrt{\frac{M+PM_0}{K}}$求出$K$。

弹簧等效质量$M_{\text{eff}}=0.0237$kg\\
$\therefore K=1615.6\pm 34.2$N/m
\end{itemize}

相对差异(与第一种方法):$10.1\%$,但两种方法的测量不确定度范围有部分重叠,结果在误差范围内基本一致;其中,后者相对不确定度较小。

\subsection{误差分析}

系统误差:
\begin{itemize}
    \item 弹簧有效质量系数$p=\frac{1}{3}$是理论近似值,实际可能略有差异;
    \item 空气阻力和弹簧内部阻尼对振动周期的影响。
\end{itemize}

随机误差:

静态测量数据离散度较大,导致不确定度较高。

\end{document}