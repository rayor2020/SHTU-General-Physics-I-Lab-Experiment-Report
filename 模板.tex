
\documentclass[12pt]{article}

% 设置中文支持
\usepackage[UTF8]{ctex}
\usepackage{amsmath}
\usepackage{graphicx}
\usepackage{booktabs}
\usepackage{multirow}
\usepackage{multicol}

% 设置页面边距
\usepackage[a4paper, left=1.91cm, right=1.91cm, top=2.54cm, bottom=2.54cm]{geometry}

% 设置标题、作者和日期
\title{物理实验报告}
\author{袁子强\hspace{1cm} 2025533009}
\date{\today}

\begin{document}

% 显示标题
\maketitle
记得剔除粗差、计算不确定度(计算过程)、图上标注原始数据、分析过程
\section{引言}
这里是引言部分的内容。

\section{实验原理}
这里是实验原理部分的内容。

\section{实验数据}
下面是一个包含合并单元格的表格示例:

\begin{table}[htbp]
    \centering
    \caption{实验数据表示例}
    \begin{tabular}{|c|c|c|c|}
        \hline
        \multirow{2}{*}{项目} & \multicolumn{2}{c|}{测量值} & \multirow{2}{*}{平均值}        \\
        \cline{2-3}
                            & 第一次                      & 第二次                  &      \\
        \hline
        物体A                 & 1.23                     & 1.25                 & 1.24 \\
        \hline
        物体B                 & 2.15                     & 2.17                 & 2.16 \\
        \hline
        物体C                 & 3.18                     & 3.20                 & 3.19 \\
        \hline
    \end{tabular}
\end{table}

\section{数据分析}
这里是数据分析部分的内容。

\section{结论}
这里是结论部分的内容。

\end{document}